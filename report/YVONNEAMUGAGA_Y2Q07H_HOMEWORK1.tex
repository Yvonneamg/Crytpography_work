\documentclass[12pt,oneside]{report}

\usepackage[utf8]{inputenc}
\usepackage[T1]{fontenc}
\usepackage{lmodern}
\usepackage{microtype}
\usepackage{amsmath,amssymb}
\usepackage{mathtools}
\usepackage{graphicx}
\usepackage{float}
\usepackage{geometry}
\usepackage{hyperref}
\usepackage{setspace}
\setstretch{1.15}

\geometry{
    a4paper,
    left=30mm,
    right=25mm,
    top=30mm,
    bottom=30mm,
}

% Images are one level above in visuals folder
\graphicspath{{../visuals/}}

\title{Cryptography and Quantum Cryptography Homework 1}
\author{Yvonne Akinyi Amugaga \\ Neptune: Y2Q07H}
\date{\today}

\begin{document}

\maketitle

% =====================================================

\section{Selected Text}

The selected Spanish four-line poem is:

\vspace{0.5em}

\begin{verse}
Bajo el cielo de España diremos ``sí, acepto'',\\
el tres de octubre sellamos nuestro amor perfecto.\\
Con el alma entrelazada y el corazón encendido,\\
me caso con el amor de mi vida, mi destino elegido.
\end{verse}


\section*{Spanish Alphabet Used}

\[
\text{A B C D E F G H I J K L M N Ñ O P Q R S T U V W X Y Z}
\]

This alphabet contains 27 letters.

% =====================================================
\section{Normalization}

All letters were converted to uppercase. Accented characters were replaced
and all non-alphabetic characters were removed.

\bigskip

\noindent\textbf{Normalized Plaintext:}

\vspace{0.5em}

\begin{verse}
BAJOELCIELODEESPANADIREMOSSIACTO\\
ELTRESDOCTUBRESELLAMOSNUESTROAMORPERFECTO\\
CONELALMAENTRELAZADAYELCORAZONENCENDIDO\\
MECASOCONELAMORDEMIVIDAMIDESTINOELEGIDO
\end{verse}

% =====================================================
\section{Caesar Cipher}

A shift of 3 was used.

\[
C = (P + 3) \mod 27
\]

Encrypted text:

\begin{verse}
EDMRHÑFLHÑRGHHVSDQDGLUHORVVLDFHSWRHÑWUHV\\
GHRFWXEUHVHÑÑDORVPXHVWURDORUSHUIHFWRFRPH\\
ÑDÑODHPWUHÑDCDGDBHÑFRUDCRPHPFHPGLGROHFDV\\
RFRPHÑDORUGHOLYLGDOLGHVWLPRHÑHJLGR
\end{verse}

% =====================================================
\section{Vigenere Cipher}

Keyword used: \textbf{AMOR}

\[
C_i = (P_i + K_i) \mod 27
\]

Encrypted text:

\begin{verse}
BMXGEWQZEWDUEPHHAZOUIDSDOEHZAÑSHTASCTDSK\\
DPDTTGPJEESCLMAGSYJVSFGGAXDJPPGWEÑIGCABV\\
LMZDAPBLRPZRZMRRYPZTODOQOYSECPBUIODDEÑOK\\
OÑDEEWODODRVMTKZDMAZDPHLIYDVLPUZDA
\end{verse}

% =====================================================
\section{Enigma Encryption}

A simplified Enigma machine was implemented in Python using:

\begin{itemize}
    \item Three rotors
    \item One reflector
    \item Rotor stepping mechanism
\end{itemize}

Encrypted text:

\begin{verse}
    BAIBIOGZFWSHINIBEZNOIWGCTCJYQXTFHHMCOIUÑ\\
    TUÑSÑZQZCPEMKDTÑQMSDQTUBÑVODHDOVUAKUBEFD\\
    DNFXIZMDÑYVSRUJMSOVNBHNMSPEBCXEPWNBPSGDU\\
    KTOCKAGBYZRRXXKWQVWPRKFHIYBYQNUÑOE
\end{verse}

% =====================================================
\section{Frequency Analysis}

Letter frequencies were computed for:

\begin{itemize}
    \item Original text
    \item Caesar cipher text
    \item Vigenere cipher text
    \item Enigma cipher text
\end{itemize}

% -----------------------------------------------------
\subsection{Original Text Statistics}

\begin{figure}[H]
    \centering
    \includegraphics[width=0.9\textwidth]{plain.png}
    \caption{Letter frequency of the original Spanish text}
\end{figure}

\textbf{Observation:}

The letters E, A, O, S, R and N appear most frequently,
which corresponds to the known statistical properties of Spanish.

% -----------------------------------------------------
\subsection{Caesar Cipher Statistics}

\begin{figure}[H]
    \centering
    \includegraphics[width=0.9\textwidth]{caesar.png}
    \caption{Letter frequency after Caesar encryption}
\end{figure}

\textbf{Observation:}

The distribution remains identical to the original but shifted cyclically.
The statistical structure of the language is still clearly recognizable.

% -----------------------------------------------------
\subsection{Vigenere Cipher Statistics}

\begin{figure}[H]
    \centering
    \includegraphics[width=0.9\textwidth]{vigenere.png}
    \caption{Letter frequency after Vigenere encryption}
\end{figure}

\textbf{Observation:}

The distribution becomes more dispersed compared to the Caesar cipher.
The original frequency peaks are partially concealed.

% -----------------------------------------------------
\subsection{Enigma Cipher Statistics}

\begin{figure}[H]
    \centering
    \includegraphics[width=0.9\textwidth]{enigma.png}
    \caption{Letter frequency after Enigma encryption}
\end{figure}

\textbf{Observation:}

The distribution appears significantly more uniform.
The characteristic Spanish frequency pattern is no longer easily recognizable,
demonstrating stronger statistical concealment.

% =====================================================
\section{Comparison of Statistical Properties}

\begin{itemize}
    \item The Caesar cipher preserves letter frequency structure.
    \item The Vigenere cipher partially hides frequency patterns.
    \item The Enigma machine produces a near-uniform distribution.
    \item Enigma encryption significantly reduces statistical predictability.
\end{itemize}

% =====================================================
\section{Sources Used}

\begin{itemize}
    \item Custom Python implementation
    \item Matplotlib library for histogram generation
    \item Course lecture notes
\end{itemize}

% =====================================================
\section{Conclusion}

This assignment demonstrates that classical substitution ciphers such as
the Caesar cipher preserve statistical properties of the language
and are vulnerable to frequency analysis.

The Vigenere cipher improves security by applying multiple shifts
based on a keyword.

The Enigma machine introduces rotor stepping and multiple layered
transformations, producing a significantly more uniform frequency
distribution and making statistical attacks substantially more difficult.

\end{document}